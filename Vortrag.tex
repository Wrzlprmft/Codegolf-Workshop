\documentclass[xcolor=dvipsnames, aspectratio=43, 14pt]{beamer}
\usefonttheme{professionalfonts}

\usepackage{xltxtra} % lädt auch fixltx2e, etex, fontspec, xunicode …

\usepackage{array, color, enumerate, fixmath, gensymb, graphicx, graphics, icomma, mathcomp, mathtools, minted, multirow, nicefrac, placeins, relsize, rotating, tabularx, tikz, units, xspace, xtab, lipsum}

\usepackage[math-style=upright]{unicode-math}

% Schriften
\setmathfont{Neo Euler}
\setromanfont[Mapping=tex-text]{Delius Unicase}
\setsansfont[Mapping=tex-text]{Delius}
\setmonofont[Scale=0.96]{Iosevka}

% Statt babel:
\usepackage{polyglossia}
\setdefaultlanguage[variant = uk]{english}

% sonstige Einstellungen
\usetheme{Rochester}
\useinnertheme{circles}
\setbeamertemplate{navigation symbols}{}

\setbeamerfont{frametitle}{family=\rmfamily}
\setbeamertemplate{blocks}[rounded]

%Presentation
\usecolortheme[accent=red]{solarized}
\setbeamercolor*{frametitle}{fg = solarized@red, bg = solarized@rebase02}
\setbeamercolor{normal text}{fg = solarized@rebase3, bg = solarized@rebase03}

\definecolor{pagenumbers}{HTML}{B4EBE7}

\addtobeamertemplate{headline}{}
{%
	\begin{tikzpicture}[remember picture,overlay]
		\node[anchor=south east] at (current page.south east) {\textcolor{pagenumbers}{\insertframenumber}};
	\end{tikzpicture}%
}

\renewcommand{\tabularxcolumn}[1]{>{\centering\arraybackslash}m{#1}}

\usemintedstyle{solarizedlight}
\newminted{python3}{
% 		fontsize = \scriptsize, 
% 		linenos, stepnumber = 5,
		frame = lines, framesep = 8pt,
		baselinestretch = 1.2,
		numbersep = 8pt,
		gobble = 0,
		bgcolor = solarized@rebase02,
		obeytabs, tabsize = 4}
% \includeonlyframes{current} %[label=current]


\begin{document}

\begin{frame}[plain]
\begin{center}

\LARGE

{\color{solarized@red}\rmfamily

Codegolf\\ mit Python}

\vfill
\normalsize

\begin{minipage}{0.8\textwidth}
	\begin{block}{Codegolf}
		Die Übung, eine Programmieraufgabe\\ mit möglichst wenig Zeichen zu lösen
	\end{block}
\end{minipage}

\vfill

Gerrit Ansmann
\end{center}
\end{frame}

\begin{frame}{Warum?}
	\begin{itemize}
		\setlength{\itemsep}{\fill}
		\item Programmiersprache entdecken
		\item mathematische Werkzeuge entdecken
		\item Algorithmen und Refaktorisierung üben
		\item dunkle Seite ausleben
		\item Spaß haben
	\end{itemize}
\end{frame}

\begin{frame}{Wie Dunkel?}
\begin{columns}
\column{0.53\linewidth}
	Scheiß auf:
	\begin{itemize}
		\item Dokumentation
		\item Lesbarkeit
		\item Nutzerfreundlichkeit
		\item sprechende Namen
		\item Code-Formatierung
		\item Speicherbedarf
		\item Laufzeit
	\end{itemize}
\column{0.47\linewidth}
\includegraphics<2>[width=\linewidth]{meme.png}
\end{columns}
 
\end{frame}


\begin{frame}{Wie?}
	\begin{itemize}
		\item Codegolfseiten, z.\,B.:\\
			codegolf.stackexchange.com
		\item mit anderen oder gegen andere\\
			(damit man auch was lernt)
		\item mit Tests
	\end{itemize}
\end{frame}

\begin{frame}{Beispiel: Fakultät}
	\begin{block}{Fakultät (explizite Definition)}
	\[f(n) = n! = \prod_{i=1}^n i = 1·2·…·n\]
	\end{block}
	
	\pause
	
	\begin{block}{Fakultät (rekursive Definition)}
	\[f(n) = n! = \begin{cases}
			n·f(n-1) & \mathrm{falls~} n \geq 1\\
			1 & \mathrm{sonst}
	          
	         \end{cases}\]
	\end{block}
\end{frame}

\begin{frame}{Aufgabe}
	Schreibe in möglichst wenig Zeichen\\ eine Python-Funktion \texttt{f} mit:
	\begin{itemize}
		\item Argument: nicht-negative ganze Zahl
		\item Rückgabe: Fakultät des Arguments
		\item Verhalten bei invalidem Argument: egal
	\end{itemize}
	
	\vfill
	
	Testfälle:\\[0.5\baselineskip]
	\setlength{\tabcolsep}{10pt}
	\renewcommand{\arraystretch}{1.3}
	\begin{tabular}{@{}c|ccccccc}
		Argument & 0 & 1 & 2 & 3 & 4 & 5 & 6 \\
		Rückgabe & 1 & 1 & 2 & 6 & 24 & 120 & 720
	\end{tabular}
\end{frame}

\begin{frame}[fragile]{Test und Zeichenzählen}
	\begin{python3code}
	from fakultaet import f
	from math import factorial
	from os import stat
	
	print(stat("fakultaet.py").st_size)
	
	for n in range(10):
	   if factorial(n)!=f(n) or type(f(n))==bool:
	      print(n, f(n), factorial(n))
	\end{python3code}
\end{frame}

\begin{frame}[fragile]{Ansatz 1: Schleifen}
	Rohfassung
	
	\vfill
	
	\begin{python3code}
	def f(n):
	   produkt = 1
	   for i in range(1,n+1):
	      produkt *= i
	   return produkt
	\end{python3code}
	
	\vfill
	
	77 Zeichen
\end{frame}

\begin{frame}[fragile]{Ansatz 1: Schleifen}
	Einbuchstabige Variable\strut
	
	\vfill
	
	\begin{python3code}
	def f(n):
	   p = 1
	   for i in range(1,n+1):
	      p *= i
	   return p
	\end{python3code}
	
	\vfill
	
	59 Zeichen
\end{frame}

\begin{frame}[fragile]{Ansatz 1: Schleifen}
	\texttt{+1} verschieben\strut
	
	\vfill
	
	\begin{python3code}
	def f(n):
	   p = 1
	   for i in range(n):
	      p *= i+1
	   return p
	\end{python3code}
	
	\vfill
	
	57 Zeichen
\end{frame}

\begin{frame}[fragile]{Ansatz 1: Schleifen}
	Unnötigen Leerraum entfernen
	
	\vfill
	
	\begin{python3code}
	def f(n):
	   p=1
	   for i in range(n):p*=i+1
	   return p
	\end{python3code}
	
	\vfill
	
	50 Zeichen
\end{frame}

\begin{frame}[fragile]{Ansatz 1.1: Reduce}
	Rohfassung
	
	\vfill
	
	\begin{python3code}
	mul = lambda x,y: x*y
	def f(n):
	   return reduce(mul, range(1,n+1), 1)
	\end{python3code}
	
	\vfill
	
	68 Zeichen
\end{frame}

\begin{frame}[fragile]{Ansatz 1.1: Reduce}
	\texttt{lambda}-Ausdruck
	
	\vfill
	
	\begin{python3code}
	mul = lambda x,y: x*y
	f = lambda n: reduce(mul, range(1,n+1), 1)
	\end{python3code}
	
	\vfill
	
	64 Zeichen
\end{frame}

\begin{frame}[fragile]{Ansatz 1.1: Reduce}
	Einsetzen und Leerzeichen entfernen
	
	\vfill
	
	\begin{python3code}
	f=lambda n:reduce(lambda x,y:x*y,range(1,n+1),1)
	\end{python3code}
	
	\vfill
	
	48 Zeichen
\end{frame}

\begin{frame}[fragile]{Ansatz 2: Rekursion}
	Rohfassung
	
	\vfill
	
	\begin{python3code}
	def f(n):
	   if n!=0:
	      return n*f(n-1)
	   else:
	      return 1
	\end{python3code}
	\vfill
	
	55 Zeichen
\end{frame}

\begin{frame}[fragile]{Ansatz 2: Rekursion}
	Ternärer Operator
	
	\vfill
	
	\begin{python3code}
	def f(n):
	   return n*f(n-1) if n!=0 else 1
	\end{python3code}
	\vfill
	
	41 Zeichen
\end{frame}

\begin{frame}[fragile]{Ansatz 2: Rekursion}
	Bedingung kürzen:
	
	\vfill
	
	\begin{python3code}
	def f(n):
	   return n*f(n-1) if n else 1
	\end{python3code}
	\vfill
	
	38 Zeichen
\end{frame}

\begin{frame}[fragile]{Ansatz 2: Rekursion}
	\texttt{lambda}-Ausdruck:
	
	\vfill
	
	\begin{python3code}
	f = lambda n: n*f(n-1) if n else 1
	\end{python3code}
	\vfill
	
	34 Zeichen
\end{frame}

\begin{frame}[fragile]{Ansatz 2: Rekursion}
	Unnötige Leerzeichen entfernen:
	
	\vfill
	
	\begin{python3code}
	f=lambda n:n*f(n-1)if n else 1
	\end{python3code}
	\vfill
	
	30 Zeichen
\end{frame}

\begin{frame}[fragile]{Ansatz 2.1: Rekursion}
	Geht noch mehr?
	
	\vfill
	
	\begin{python3code}
	def f(n):
	   if int(n==0):
	      return int(n==0)
	   else:
	      return n*f(n-1)
	\end{python3code}
	
	\vfill
	
	68 Zeichen
\end{frame}

\begin{frame}[fragile]{Ansatz 2.1: Rekursion}
	Logische Operatoren geben\\ das erste entscheidende Argument zurück.
	
	\vfill
	
	\begin{python3code}
	def f(n):
	   return int(n==0) or n*f(n-1)
	\end{python3code}
	
	\vfill
	
	39 Zeichen
\end{frame}

\begin{frame}[fragile]{Ansatz 2.1: Rekursion}
	\texttt{lambda}-Ausdruck
	
	\vfill
	
	\begin{python3code}
	f = lambda n: int(n==0) or n*f(n-1)
	\end{python3code}
	
	\vfill
	
	35 Zeichen
\end{frame}

\begin{frame}[fragile]{Ansatz 2.1: Rekursion}
	Ersetze \texttt{int(n==0)} durch Arithmetik:
	
	\vfill
	
	\begin{python3code}
	f = lambda n: 0**n or n*f(n-1)
	\end{python3code}
	
	\vfill
	
	30 Zeichen
\end{frame}

\begin{frame}[fragile]{Ansatz 2.1: Rekursion}
	Entferne Leerzeichen:
	
	\vfill
	
	\begin{python3code}
	f=lambda n:0**n or n*f(n-1)
	\end{python3code}
	
	\vfill
	
	27 Zeichen
\end{frame}

\begin{frame}[fragile]{Ansatz 3: Mogeln}
	\begin{python3code}
	import math
	f=math.factorial
	\end{python3code}
	
	\vfill
	
	28 Zeichen
\end{frame}

\begin{frame}[fragile]{Trick:\\ Ternären Operator kürzen}
	Sei \texttt{b} ein Boolean, \texttt{0} oder \texttt{1}.
	
	\vfill
	
	\begin{python3code}
	x=a if b else c
	\end{python3code}
	
	\vfill
	
	kann gekürzt werden zu:
	
	\vfill
	
	\begin{python3code}
	x=[c,a][b]
	\end{python3code}
\end{frame}

\begin{frame}[fragile]{Trick:\\ Billiger Switch-Ausdruck}
	\begin{python3code}
	if p==0:x="a"
	if p==1:x="b"
	if p==2:x="c"
	if p==3:x="d"
	\end{python3code}
	
	\vfill
	
	kann gekürzt werden zu:
	
	\vfill
	
	\begin{python3code}
	x="abcd"[p]
	\end{python3code}
\end{frame}

\begin{frame}[fragile]{Trick:\\ Vergleiche verketten}
	\begin{python3code}
	if a<2 and b>1:
	\end{python3code}
	
	\vfill
	
	kann gekürzt werden zu:
	
	\vfill
	
	\begin{python3code}
	if a<2>1<b:
	\end{python3code}
\end{frame}

\begin{frame}[fragile]{Trick:\\ Vergleiche verketten 2}
	(nur Python 2)
	
	\begin{python3code}
	if a<b and c<d:
	\end{python3code}
	
	\vfill
	
	kann gekürzt werden zu:
	
	\vfill
	
	\begin{python3code}
	if a<2>1<b:
	\end{python3code}
\end{frame}



\end{document}