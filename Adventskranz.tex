\documentclass[a4paper, 14 pt]{extarticle}
 
\usepackage{amsmath, amssymb}

\usepackage{xltxtra}

\usepackage{array, color, enumerate, fixmath, gensymb, graphicx, graphics, icomma, mathcomp, mathtools, multirow, nicefrac, placeins, relsize, rotating, tabularx, titlesec, units, xspace, xtab, lipsum}

\usepackage[math-style=upright]{unicode-math}



% ----Schriften----
\setmathfont{Neo Euler}
\setsansfont[Mapping=tex-text]{Delius Unicase}
\setromanfont[Mapping=tex-text]{Delius}
\setmonofont[Scale=0.96]{Iosevka}
\newfontfamily\swfamily{Delius Swash Caps}

% ----Sprache----
\usepackage{polyglossia}
\setdefaultlanguage[spelling=new, latesthyphen=true]{german}


% ----Seite einstellen----
\setlength{\columnsep}{1 cm}
\setlength{\textwidth}{17 cm}
\setlength{\evensidemargin}{2 cm} %Abstand Text – äußerer Rand

\setlength{\hoffset}{-1 in}
\setlength{\oddsidemargin}{21 cm}
\addtolength{\oddsidemargin}{-\textwidth}
\addtolength{\oddsidemargin}{-\evensidemargin}

\setlength{\voffset}{-1 in}
\setlength{\topmargin}{1.6 cm}

\setlength{\headsep}{0 pt}
\setlength{\headheight}{0 pt}
\setlength{\textheight}{29.7 cm}
\addtolength{\textheight}{-2 \topmargin}
\addtolength{\textheight}{- \headheight}
\addtolength{\textheight}{- \headsep}
%\addtolength{\textheight}{- \footskip}

\hyphenpenalty=500
\pretolerance=150
\tolerance=1500
\setlength{\emergencystretch}{\textwidth}

\setlength{\parindent}{0pt} 

\titleformat*{\section}{\sffamily\larger[2]}
\titleformat*{\subsection}{\swfamily\larger[2]}

\setcounter{secnumdepth}{2}

\thispagestyle{empty}


\setcounter{section}{1}

\begin{document}
\section{Der ausgewuchtete\\ Adventskranz}

Schreibe eine Funktion, die erlaubt, beliebige Kerzen auf einem Adventskranz zu so zu verteilen, dass dieser keine Schlagseite hat.

\subsection{Eingabe}

Eine beliebige Iterable aus positiven Zahlen.
Jede Zahl steht für eine Kerze und gibt deren Masse an.

\subsection{Rückgabe}

Eine Iterable mit Positionen der Kerzen (in der Ebene des Adventskranzes).
Werden die Kerzen auf diese Positionen gesetzt, muss der Schwerpunkt des Adventskranzes (mit angemessener Genauigkeit) der Mitte des Adventskranzes entsprechen.
Die Reihenfolge der Kerzen in Ein- und Ausgabe ist dieselbe.

\subsection{Unlösbare Eingaben}
Wenn die Eingabe einem nicht lösbaren Problem entspricht (z.\,B. wenn eine Kerze schwerer ist als alle anderen zusammen), ist es egal, wie sich die Funktion verhält.

\subsection{Annahmen}
\begin{itemize}
	\item Der Adventskranz sei homogen und zweidimensional.
	\item Der Adventskranz habe Radius~1.
	\item Die Kerzen seien punktförmig.
	\item Die Kerzen werden nicht angezündet\\ (werden also nicht langsam leichter).
	\item Die Kuh sei eine Kugel.
\end{itemize}



\end{document}
