\documentclass[a4paper, 14 pt]{extarticle}
 
\usepackage{amsmath, amssymb}

\usepackage{xltxtra}

\usepackage{array, color, enumerate, fixmath, gensymb, graphicx, graphics, icomma, mathcomp, mathtools, minted, multirow, nicefrac, placeins, relsize, rotating, tabularx, titlesec, units, xspace, xtab, lipsum}

\usepackage[math-style=upright]{unicode-math}



% ----Schriften----
\setmathfont{Neo Euler}
\setsansfont[Mapping=tex-text]{Delius Unicase}
\setromanfont[Mapping=tex-text]{Delius}
\setmonofont[Scale=0.96]{Iosevka}
\newfontfamily\swfamily{Delius Swash Caps}

% ----Sprache----
\usepackage{polyglossia}
\setdefaultlanguage[spelling=new, latesthyphen=true]{german}


% ----Seite einstellen----
\setlength{\columnsep}{1 cm}
\setlength{\textwidth}{17 cm}
\setlength{\evensidemargin}{2 cm} %Abstand Text – äußerer Rand

\setlength{\hoffset}{-1 in}
\setlength{\oddsidemargin}{21 cm}
\addtolength{\oddsidemargin}{-\textwidth}
\addtolength{\oddsidemargin}{-\evensidemargin}

\setlength{\voffset}{-1 in}
\setlength{\topmargin}{1.6 cm}

\setlength{\headsep}{0 pt}
\setlength{\headheight}{0 pt}
\setlength{\textheight}{29.7 cm}
\addtolength{\textheight}{-2 \topmargin}
\addtolength{\textheight}{- \headheight}
\addtolength{\textheight}{- \headsep}
%\addtolength{\textheight}{- \footskip}

\hyphenpenalty=500
\pretolerance=150
\tolerance=1500
\setlength{\emergencystretch}{\textwidth}

\setlength{\parindent}{0pt} 

\titleformat*{\section}{\sffamily\larger[2]}
\titleformat*{\subsection}{\swfamily\larger[2]}

\setcounter{secnumdepth}{2}

\thispagestyle{empty}

\usemintedstyle{bw}
\newminted{python3}{
% 		fontsize = \scriptsize, 
% 		linenos, stepnumber = 5,
		frame = lines,
		framerule = 1pt,
		framesep = 8pt,
		baselinestretch = 1.2,
		numbersep = 8pt,
		gobble = 0,
		obeytabs, tabsize = 4}

\setcounter{section}{1}

\begin{document}
\section{Überüberüberüberüber-überüberüberübermorgen, \\ Kinder, wird’s was geben}

Gib die erste Strophe von »Morgen, Kinder, wird’s was geben« aus, und zwar mit korrekten Zeitangaben.

\subsection{Eingabe}

Keine.

\subsection{Ausgabe}

Am 24. Dezember soll ausgegeben werden:

\begin{quote}
Heute, Kinder, wird’s was geben,\\
heute werden wir uns freun;\\
welch ein Jubel, welch ein Leben\\
wird in unserm Hause sein!\\
Keinmal werden wir noch wach;\\
heisa, dann ist Weihnachtstag!
\end{quote}

Am 23. Dezember soll ausgegeben werden (ohne eckige Klammern):

\begin{quote}
[Morgen], Kinder, wird’s was geben,\\{}
[morgen] werden wir uns freun;\\
welch ein Jubel, welch ein Leben\\
wird in unserm Hause sein!\\{}
[1]-mal werden wir noch wach;\\
heisa, dann ist Weihnachtstag!
\end{quote}

An allen anderen Tagen müssen die in eckigen Klammern stehenden Wörter und Zahlen entsprechend angepasst ausgegeben werden, wobei »Morgen« durch »Übermorgen«, »Überübermorgen«, usw. zu ersetzen ist.

\end{document}
